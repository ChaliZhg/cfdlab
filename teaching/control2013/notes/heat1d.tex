\documentclass[12pt]{article}
\usepackage[width=16cm]{geometry}                % See geometry.pdf to learn the layout options. There are lots.
\geometry{a4paper}                   % ... or a4paper or a5paper or ... 
%\geometry{landscape}                % Activate for for rotated page geometry
%\usepackage[parfill]{parskip}    % Activate to begin paragraphs with an empty line rather than an indent
\usepackage{graphicx}
\usepackage{amssymb}
\usepackage{amsmath}
\usepackage{aliases}
\usepackage{color}
\usepackage{url}

\usepackage{listings}
\usepackage{cancel}
\usepackage{textcomp}

\lstset{
   language=matlab,
   keywordstyle=\bfseries\ttfamily\color[rgb]{0,0,1},
   identifierstyle=\ttfamily,
   commentstyle=\color[rgb]{0.133,0.545,0.133},
   stringstyle=\ttfamily\color[rgb]{0.627,0.126,0.941},
   showstringspaces=false,
   basicstyle=\small,
   numberstyle=\footnotesize,
   numbers=none,
   stepnumber=1,
   numbersep=10pt,
   tabsize=2,
   breaklines=true,
   prebreak = \raisebox{0ex}[0ex][0ex]{\ensuremath{\hookleftarrow}},
   breakatwhitespace=false,
   aboveskip={0.1\baselineskip},
    columns=fixed,
    upquote=true,
    extendedchars=true,
% frame=single,
    backgroundcolor=\color[rgb]{0.9,0.9,0.9}
}

\title{One dimensional heat equation}
\author{Deep Ray, Ritesh Kumar, Praveen. C, Mythily Ramaswamy, J.-P. Raymond}
\date{}

\begin{document}



\maketitle
\begin{center}
IFCAM Summer School on Numerics and Control of PDE\\
22 July - 2 August 2013\\
IISc, Bangalore\\
\url{http://praveen.cfdlab.net/teaching/control2013}
\end{center}

%----------------------------------------------------------------------------------------

\section{The model}
The shifted 1D heat equation is given by
\[
z_t = \mu z_{xx} + \alpha z, \quad x \in (0,1) \times (0,T)
\]
with boundary conditions
\[
z(0,t) = 0, \quad z(1,t) = u(t)
\]
and initial condition
\[
z(x,0) = z_0(x) , \quad x \in (0,1)
\]
Here $\alpha \ge 0$ and $\mu > 0$. 
%----------------------------------------------------------------------------------------

\subsection{Weak formulation}
We assume $z_0 \in L^2(0,1)$. We wish to find $z \in L^2(0,T;H^1_{\{0\}}(0,1))$ such that
\[
 \dd{}{t}(z(t), \phi)_{L^2} = - \mu \int \limits^{1}_{0} \df{z}{x} \dd{\phi}{x} \, dx +  \alpha \int \limits^{1}_{0} z(x,t) \phi \, dx, \quad \forall \phi \in H^1_0(0,1)
\]
\[
 z(1,t) = u(t)
\]
\[
 (z(0),\phi)_{L^2} = (z_0 ,\phi)_{L^2}
\]
%----------------------------------------------------------------------------------------

\subsection{FEM approximation}
Consider a regular subdivision of $[0,1]$
\[
 0 = x_0 < x_1 < ... < x_N = 1, \quad h = \frac{1}{N}, \quad x_k = kh \quad \forall k \in \{0,1,...,N\}
\]
We work with the finite dimensional subspaces of $H^1_{0}(0,1)$
\begin{equation*}
\begin{aligned}
  & Z_h = \{ \phi \in C([0,1]) : \phi|_{[x_{i-1},x_i]} \in P_1, \quad \phi(0) = 0 \} \\
  & Z_{h,0} = \{ \phi \in C([0,1]) : \phi|_{[x_{i-1},x_i]} \in P_1, \quad \phi(0) = \phi(1) = 0 \}
\end{aligned}
\end{equation*}
We denote the standard $P_1$ bases for $Z_{h,0}$ and $Z_{h}$ by $\{\phi_1, \phi_2,...,\phi_{N-1}\}$ and $\{\phi_1, \phi_2,...,\phi_N\}$ respectively, with $\phi_j(x_i) = \delta_{ij}$. The test function are chosen from $Z_{h,0}$. The approximate weak formulation is given as
\[
 \text{Find } z = \sum \limits_{i=1}^{N-1} z_i \phi_i + u(t) \phi_N \in Z_h, \quad \text{ such that}
\]
\[
 \dd{}{t}(z(t), \phi_k)_{L^2} = - \mu \int \limits^{1}_{0} \df{z}{x} \dd{\phi_k}{x} \, dx +  \alpha \int \limits^{1}_{0} z(x,t) \phi_k \, dx, \quad \forall k \in \{1,..., N-1\}
\]
\[
 (z(0),\phi_k)_{L^2} = (z_0 ,\phi_k)_{L^2}, \quad \forall k \in \{1,..., N-1\}
\]
We obtain a a differential system 
\[
 \M \dd{\z}{t} = \A_\alpha \z + \B u , \quad \M \z(0) = ((z_0,\phi_i)_{L^2})_{1\le i \le {N-1}}
\]
where
\begin{itemize}
 \item $\z = (z_1,z_2,...,z_{N-1})^\top$
 \item $\M \in \re^{(N-1) \times (N-1)}, \quad M_{i,j} = \int \limits_0^1 \phi_i \phi_j \, dx$
 \item $\A \in \re^{(N-1) \times (N-1)}, \quad A_{i,j} = - \mu \int \limits_0^1 \dd{\phi_i}{x} \dd{\phi_j}{x} \, dx$
 \item $\A_\alpha = \A + \alpha \M$
 \item $\B =  \frac{\mu}{h} (0,0,...,0,1)^\top \in \re^{N-1}$
 \item $u \in \re$
\end{itemize}
Using exact integration for $\M$ and $\A$ gives
\[
 \M = h \left[\begin{array}{ccccc}
         \frac{2}{3} & \frac{1}{6} & 0 & \cdots & 0 \\
         \frac{1}{6} & \frac{2}{3} & \frac{1}{6}  & \vdots & 0 \\
         0 & \ddots  & \ddots & \ddots & 0 \\
         0 & \cdots & \frac{1}{6} & \frac{2}{3} & \frac{1}{6} \\
         0 & \cdots & 0 & \frac{1}{6} & \frac{2}{3}
        \end{array} \right] , \quad
 \A = \frac{\mu}{h} \left[\begin{array}{ccccc}
         \-2 & 1 & 0 & \cdots & 0 \\
         1 & -2 & 1  & \vdots & 0 \\
         0 & \ddots  & \ddots & \ddots & 0 \\
         0 & \cdots & 1 & -2 & 1 \\
         0 & \cdots & 0 & 1 & -2
        \end{array} \right]       
\]
We use the trapezoidal rule for $u'(t) \int_0^1 \phi_N \phi_{N-1}$ which will evaluate out as zero. 

\vspace{2mm}

\noindent
The matrices are computed in the matlab code {\tt matrix\_fem.m}
%----------------------------------------------------------------------------------------
\section{Time integration using BDF}
Assume for now that the feedback control is of the form 
\[
 u(t) = - \K \z(t), \quad \K \in \re^{1\times(N-1)}
\]
We shall use a Backward Differentiation Formula (BDF) for time integration. For the first time step we use BDF1 (backward Euler scheme)
\[
 \M \frac{\z^1 - \z^0}{\Delta t} = (\A_\alpha - \B \K)\z^1
\]
or
\[
 \left[ \frac{\M }{\Delta t} - (\A_\alpha - \B \K) \right] \z^1 =  \frac{\M}{\Delta t} \z^0 
\]
For the remaining time steps we use BDF2
\[
 \M \frac{\z^{n+1} - \frac{4}{3} \z^{n} + \frac{1}{3} \z^{n-1}}{\frac{2}{3}\Delta t} = (\A_\alpha - \B \K)\z^1
\]
or
\[
 \left[ \frac{\M }{\frac{2}{3}\Delta t} - (\A_\alpha - \B \K) \right] \z^{n+1} =  -\frac{\M}{\frac{2}{3}\Delta t}\left( -\frac{4}{3} \z^{n} + \frac{1}{3} \z^{n-1} \right)
\]

%--------------------------------------------------------------------------------
\section{Excercises}
\begin{enumerate}
\item The code in {\tt heat.m} solves the problem with homogeneous Dirichlet boundary conditions at $x=0$ and $x=1$ and initial condition
\[
z(x,0) = x^2 (1-x)^3
\]
Since $z(1,t) = u(t) = 0$ it is enough to solve
\[
\M \dd{\z}{t} = \A \z
\]
Run the code {\tt heat.m} with $\alpha =0$
\begin{lstlisting}
alpha = 0; 
mu = 1;
\end{lstlisting}
Are all the eigenvalues stable ? How do the solution and energy behave with time ?

\item Evaluate the decay rate of energy for initial condition $z(x,0) = \sin(\pi x)$
\begin{lstlisting}
P = polyfit(t, log(energy'), 1)
P(1) % Exact rate = 2*mu*pi^2
\end{lstlisting}
      
\item With initial condition $z(x,0) = \sin(\pi x)$ run {\tt heat.m} with
\begin{lstlisting}
alpha = 0.4 + pi^2 * mu
\end{lstlisting}
Are all the eigenvalues stable ? How do the solution and energy behave with time ?

\item For same $\alpha$ as above, compute 10 eigenvalues with largest real part using {\tt eigs} function. For each unstable eigenvalue $\lambda$, compute eigenvector $\V$ of $(\A_\alpha^\top,\M^\top)$
\[
 \A_\alpha^\top \V = \lambda \M^\top \V
\]
and check if 
\[
 \B^\top \V \ne 0
\]
If the above is true for all unstable eigenvalues, then the system is stabilizable. Is the heat equation system stabilizable ?

\item Now modify {\tt heat.m} to implement the following non-homogeneous Dirichlet boundary condition
\[
z(1,t) = u(t) = 1
\]
and use the same initial condition as before. Use $\mu=1$ and $\alpha=0$. The solution should converge to
\[
\lim_{t \to \infty} z(x,t) = x
\]

\end{enumerate}

%--------------------------------------------------------------------------------
\section{Minimal norm feedback control}
The minimal norm control is given by
\[
u(t) = - \K z (t)
\]
The feedback matrix $\K$ is given by
\[
\K = \R^{-1} \B^\top \bdP \M
\]
where $\bdP$ is solution of algebraic Riccati equation (ARE)
\[
\A^\top_\alpha \bdP \M + \M^\top \bdP \A_\alpha - \M^\top \bdP \B \R^{-1} \B^\top \bdP \M  = 0
\]
This can be solved using the {\tt care} function
\begin{lstlisting}
[P,L,K] = care(A,B,R,Q,S,M);
\end{lstlisting}
which solves the more general ARE
\[
\A^\top \bdP \M + \M^\top \bdP \A - (\M^\top \bdP \B \ + \bdS^\top)  \R^{-1} (\B^\top \bdP \M +\bdS) + \Q = 0
\]
where $\bL$ gives the eigenvalues of $(\A- \B \K,\M)$. Since $\bdS=0$ for our problem, we can simply replace it with {\tt []} while calling the {\tt care} function. Furthermore, for the minimal norm feedback, we take $\Q=0$, which can be set in {\tt matrix\_fem.m}
%--------------------------------------------------------------------------------
\section{Excercises}

\begin{enumerate}
\item Set $\Q=0$ in {\tt matrix\_fem.m} and use
\begin{lstlisting}
alpha = 0.4 + pi^2 * mu
\end{lstlisting}
Run the code {\tt heat\_lqr.m}. Use initial condition $z(x,0) = \sin(\pi x)$. Observe the modification in the eigenvalues; zoom the eigenvalue plot near the origin and locate the unstable eigenvalue and modified eigenvalue; note that the unstable eigenvalue is reflected about the imaginary axis. What about the remaining stable eigenvalues? Is the solution of the heat equation stable ?

\item Save the feedback matrix evaluated
\begin{lstlisting}
save('feedback.mat','K')
\end{lstlisting}
Load $\K$ into {\tt heat.m}, and for {\tt alpha = 0} (no shift) introduce control using the feedback matrix
\begin{lstlisting}
load('feedback');
A = A-B*sparse(K);
\end{lstlisting}
Find the decay rate in this case using {\tt polyfit} function. Has it improved? We expect it to be atleast $2(0.4 + \mu \pi^2)$.

\end{enumerate}

%--------------------------------------------------------------------------------

\section{Feedback control using LQR approach}
Measurement
\[
y_m = C z, \qquad \textrm{e.g.} \qquad C = I
\]
Performance measure
\begin{eqnarray*}
J &=& \frac{1}{2}\int_0^\infty y_m^\top Q_m y_m \ud t + \frac{1}{2} \int_0^\infty u^\top R 
u \ud t \\
&=& \frac{1}{2}\int_0^\infty z^\top Q z \ud t + \frac{1}{2} \int_0^\infty u^\top R u \ud   
t , \qquad Q = C^\top Q_m C
\end{eqnarray*}
Find feedback law
\[
u = -\K \z
\]
which minimizes $J$. The feedback matrix $\K$ is given by
\[
\K = \R^{-1} \B^\top \bdP \M
\]
where $\bdP$ is solution of algebraic Riccati equation (ARE)
\[
\A^\top_\alpha \bdP \M + \M^\top \bdP \A_\alpha - \M^\top \bdP \B \R^{-1} \B^\top \bdP \M + \Q = 0
\]
%The solution exists if $(A,B)$ is controllable and $(A,C)$ is observable.
%--------------------------------------------------------------------------------

\section{Excercises}

\begin{enumerate}

\item In {\tt matrix\_fem.m}, put $\Q = \M$ (which corresponds to $Q = I$). Run {\tt heat\_lqr.m}. How do the eigenvalues change this time? What is the behaviour of the solution?

\item Plot the intial condition and the numerical solution after one time step. How does it differ from the previous Bernoulli control case?

\item How does the solution and control vary as {\tt mu} is decreased?

\item In {\tt matrix\_fem.m}, vary the value of $\R$ in (0.01, 50) (choose 6-7 values in the range). How does $\K$ vary with $\R$?

\item As before, save the feedback matrix and load it into {\tt heat.m}. Is the decay rate as expected?
\end{enumerate}

%--------------------------------------------------------------------------------

\section{Control based on stabilizing only the unstable components}
Let the us denote by $\lambda_i$ and $\V_i$ the eigenvalues and normalized eigenvectors such that
\[
 \A \V_i = \lambda_i \M \V_i
\]
We assume that $\alpha$ is chosen such that
\[
 \lambda_{N-1} < .... <\lambda_{N_\alpha +1} < -\alpha < \lambda_{N_\alpha} < ... < \lambda_1
\]
Thus in the shifted system for the heat equation, there will be $N_\alpha$ unstable eigenvalues.
The eigenvectors $\V_i$ form a basis for $\re^{N-1}$ and have the property
\[
 \V_i^\top \M \V_j = \delta_{i,j}, \quad \V_i^\top \A \V_j = \delta_{i,j} \lambda_i
\]
Let $\Sigma \in \re^{(N-1) \times (N-1)}$ with $\V_i$ forming the columns, and consider the variable change
\[
 \z = \Sigma \zz
\]
Thus the shifted system can be written as
\[
 \dd{\zz}{t} = \Lambda_\alpha \zz + \BB u
\]
where
\[
 \Lambda_\alpha = \begin{bmatrix}
                   \alpha + \lambda_1 & 0 & \cdots & 0\\
                   0 & \alpha + \lambda_2 & \cdots & 0\\
                   \vdots & \vdots & \ddots & &\\
                   0 & 0 & 0 & \alpha + \lambda_{N-1}
                  \end{bmatrix}, \qquad
 \left[\BB\right]_i = V^\top_i \B, \quad \forall 1 \le i \le N-1                                   
\]
Next we project the system onto the unstable subspace using the projection operator $\PIu \in \LL(\re^{N-1},\re^{N_\alpha})$
\[
 \dd{\zzu}{t} = \Lambda_{\alpha,u} \zzu + \BBu u
\]
where
\[
 \Lambda_{\alpha,u} = \begin{bmatrix}
                   \alpha + \lambda_1 & 0 & \cdots & 0\\
                   0 & \alpha + \lambda_2 & \cdots & 0\\
                   \vdots & \vdots & \ddots & &\\
                   0 & 0 & 0 & \alpha + \lambda_{N_\alpha}
                  \end{bmatrix}, \qquad
 \left[\BBu \right]_i = V^\top_i \B, \quad \forall 1 \le i \le N_\alpha                                   
\]
and $\zzu = \PIu \zz$ are the first $N_\alpha$ components. We find the feedback matrix for this reduced system, by solving the Bernoulli equation
\[
 \PPu \Lambda_{\alpha,u} + \Lambda^\top_{\alpha,u} \PPu - \PPu \BB \BB^\top \PPu = 0
\]
\[
 \Lambda_{\alpha,u} - \BBu \BBu^\top \PPu \text{ is stable}
\]
The corresponding matrix $\bdP \in \LL(\re^{N-1})$ such that $(\M,\A_\alpha - \B\B^\top\bdP\M)$ is stable is given by
\[
 \bdP = \Sigma_u \PPu \Sigma^\top_u
\]
where $\Sigma_u$ is the matrix of eigenvectors corresponding to the unstable eigenvalues. This is implemented in {\tt heat\_lqru.m}
%----------------------------------------------------------------------------------------

\paragraph{Excercises}

\begin{enumerate}

\item In {\tt matrix\_fem.m}, put $\Q = 0$. Run {\tt heat\_lqru.m}. How do the eigenvalues change this time? What is the behaviour of the solution?

\item Plot the intial condition and the numerical solution after one time step. How does it differ from the previous two kinds of control?

\item For the same value of {\tt alpha}, which of the three types of controls discussed so far performs the best?

\item Lecture 5 has the exact solutions for this problem, with {\tt alpha = 10}. Take 
\begin{lstlisting}
 z0 = sqrt(2)*sin(pi* x) 
\end{lstlisting}
in the equation and find the expressions for exact control and solution. Plot and compare 
\begin{itemize}
 \item evolution of the numerical control and the exact control 
 \item numerical and exact solutions at the final time
\end{itemize}

\item Save the feedback matrix and load it into {\tt heat.m}. Is the decay rate as expected?
\end{enumerate}

%-----------------------------------------------------------------------------------

\section{System with noise and partial information}
Consider the system with noise in the model and initial condition
\[
z_t = \mu z_{xx} + \alpha z + w, \qquad z(x,0) = z_0(x) + \eta
\]
where $w$ and $\eta$ are error/noise terms. The boundary conditions are as before
\[
z(0,t) = 0, \quad z(1,t) = u(t)
\]
We may not have access to the full state but only some partial information which is also corrupted by noise.
\[
y = Hz  + v
\]
where $H$ is a suitable measure. We shall consider the case where $H$ is given by
\[
 Hz(t) = z_x(0,t)
\]
In the FEM setup, we get the system
\[
 \M \dd{\z}{t} = \A_\alpha \z + \B u + \w , \quad \M \z(0) = ((z_0,\phi_i)_{L^2})_{1\le i \le {N-1}}
\]
\[
 \y = \bH \z + \bdv
\]
where 
\[
 \bH = \frac{\mu}{h}(2,0.5,0,...,0)^\top \in \re^{N-1}
\]


%--------------------------------------------------------------------------------

\subsection{Estimation problem}
Linear estimator
\[
\dd{z_e}{t} = A_\alpha z_e + B u + L(y - H z_e)
\]
We determine the filtering gain $L$ by minimizing
\[
J = \frac{1}{2} \int_0^\infty (y - H z_e)^\top R^{-1}_v (y - H z_e) \ud t +    \frac{1}{2} \int_0^\infty w^\top R_w^{-1} w \ud t
\]
The solution is given by
\[
L = - P_e H^\top R^{-1}_v
\]
where $P_e$ is solution of
\[
A_\alpha P_e + P_e A_\alpha^\top - P_e H^\top R_v^{-1}  H P_e + R_w = 0
\]
The operators $R_w$ and $R_v$ are chosen according to the apriori knowledge we have on the model noise and the measurement noise. In the FEM setup, we have
\[
\M \dd{\z_e}{t} = \A_\alpha \z_e + \B u + \bL(\y - \bH \z_e)
\]
\[
\bL = - \M \bdP_e \bH^\top \R^{-1}_\bdv
\]
where $\bdP_e$ is solution of
\[
\A_\alpha \bdP_e \M + \M \bdP_e \A_\alpha^\top - \M \bdP_e \bH^\top \R_\bdv^{-1}  \bH \bdP_e \M + \R_\w = 0
\]
\[
 (\M,\A_\alpha - \bL \bH) \text{ is stable} 
\]

%--------------------------------------------------------------------------------

\subsection{Coupled linear system}
The feedback is based on estimated solution $u = -\K \z_e$
\begin{eqnarray*}
\M \dd{\z}{t} &=& \A_\alpha \z - \B \K \z_e + \w \\
\M \dd{\z_e}{t} &=& \bL \bH \z + (\A_\alpha - \B \K - \bL \bH) \z_e + \bL \bdv
\end{eqnarray*}
or in matrix form
\[\begin{bmatrix}
   \M & 0\\
   0 & \M \\
  \end{bmatrix}
\dd{}{t} \begin{bmatrix}
\z \\
\z_e \end{bmatrix} = \begin{bmatrix}
\A_\alpha & -\B \K \\
\bL \bH & \A_\alpha - \B \K - \bL \bH \end{bmatrix} \begin{bmatrix}
\z \\ \z_e \end{bmatrix} + \begin{bmatrix}
\I & 0 \\
0 & \bL \end{bmatrix} \begin{bmatrix}
\w \\ \bdv \end{bmatrix}
\]
The initial condition is given by
\[
\z(0) = \z_0 + \eta, \qquad \z_e(0) = \z_0
\]
This is implemented in program {\tt heat\_est.m}
%----------------------------------------------------------------------------------------

\paragraph{Excercises}

\begin{enumerate}

\item Run program {\tt heat\_est.m}

\item Plot the eigenvalues of the coupled system. Are they stable?

\item The matrices $\K$ and $\bL$ should improve estimation and decay rate of the unshifted problem. Check this via the following steps
   \begin{itemize}
    \item Save the matrices $\K$ and $\bL$
    
    \item Set {\tt alpha = 0} and evaluate the decay rate of $\z$ with $\K = 0$ and $\bL=0$. 
    
    \item Load the saved matrices and evaluate the decay rate. Has it improved?
   \end{itemize}


\end{enumerate}

%--------------------------------------------------------------------------------
\subsection{Stabilizing the unstable component}
As done before, the feedback gain matrix $\K$ can be evaluated by considering the projected system
\[
 \dd{\zzu}{t} = \Lambda_{\alpha,u} \zzu + \BBu u
\]
We use the variable change
\[
 \z = \Sigma \zz
\]
and define the operator $\HH$ by
\[
 \HH \zz = \bH \z
\]
We shall work with the projected variable $\zzu = \PIu \zz$, and thus define the corresponding measure operator
\[
 \HHu = \HH \PIu
\]
The filtering gain $\PP_e$ for the reduced projected system is obtained as the solution of the ARE
\[
 \PP_e \Lambda_{\alpha,u}^\top + \Lambda_{\alpha,u} \PP_e - \PP_e \HHu^\top \R_\bdv^{-1} \HHu \PP_e + \QQ_\w
\]
\[
 \Lambda_{\alpha,u} - \PP_e \HHu^\top \R_\bdv^{-1} \HHu \text{ is stable} 
\]
where $\QQ_\w = \Sigma_u^\top \R_\w \Sigma_u$. The corresponding $\bdP_e$ such that $(\M,\A_\alpha - \M \bdP_e \bH^\top \R_\bdv^{-1} \bH)$ is stable is given by
\[
 \bdP_e = \Sigma_u \PP_e \Sigma^\top_u
\]

%--------------------------------------------------------------------------------
\paragraph{Excercises}

\begin{enumerate}

\item Write a code {\tt heat\_estu.m} for the above via the following steps
\begin{itemize}
\item Copy the contents of {\tt heat\_est.m}

\item Evaluate the feedback gain $\K$ as done in {\tt heat\_lqru.m}

\item Evaluate the projected matrix $\HHu$ 
\begin{lstlisting}
Hu = full(H)*V; 
\end{lstlisting}
where {\tt V} is the matrix of unstable eigenvectors ( you would have already evaluated this while obtaining $\K$ )

\item Evaluate $\QQ_\w$ 
\begin{lstlisting}
Rwu = V'*full(Rw)*V; 
\end{lstlisting}

\item Solve the ARE for the reduced system to obtain $\PP_e$
\begin{lstlisting}
Peu = care(D',full(Hu)',full(Rwu),full(Rv));
\end{lstlisting}
where {\tt D} is the diagonalized matrix of the reduced system ( also evaluated while finding $\K$ )

\item Find $\bdP_e$ for the original system
\begin{lstlisting}
Pe = V*Peu*V'; 
\end{lstlisting}

\item Finally evaluate $\bL$ by
\begin{lstlisting}
L = M*Pe*H'/Rv; 
\end{lstlisting}

\item Run {\tt heat\_estu.m} with {\tt alpha = 10}. How do the solution and energy behave?

\end{itemize}

\item Check whether the $\K$ and $\bL$ improve the estimation and decay rate of the unshifted problem 

\begin{itemize}
\item Save the matrices $\K$ and $\bL$
    
\item Set {\tt alpha = 0} and evaluate the decay rate of $\z$ with $\K = 0$ and $\bL=0$. It would be easier to copy the code of {\tt heat\_estu.m} into a new file, say {\tt heat\_estu0.m} and remove all evaluations of $\bdP$ and $\bdP_e$.
    
\item Load the saved matrices and evaluate the decay rate. Has it improved?
\end{itemize}

\end{enumerate}

%--------------------------------------------------------------------------------

\section{List of Programs}

\begin{enumerate}

\item {\tt matrix\_fem.m}: Computes FEM matrices

\item {\tt hautus.m}: Checks the stabilizability of the system using Hautus criterion

\item {\tt feedback\_matrix.m}: Computes feedback matrix

\item {\tt heat.m}: Solves the heat equation for a given $\alpha$, without feedback

\item {\tt heat\_lqr.m}: Solves the heat equation for a given $\alpha$, with feedback

\item {\tt heat\_lqr.m}: Solves the heat equation for a given $\alpha$, stabilizing only the unstable components

\item {\tt heat\_est.m}: Solves the coupled estimation and control problem for the heat equation for a given $\alpha$

\item {\tt heat\_estu.m}: Solves the coupled estimation and control problem for the heat equation for a given $\alpha$, stablilizing only the unstable components

\item {\tt heat\_estu0.m}: Solves the coupled estimation and control problem for the heat equation for $\alpha = 0$, using matrice from {\tt heat\_estu.m}

\end{enumerate}

\end{document}  
