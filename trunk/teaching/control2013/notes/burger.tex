\documentclass[12pt]{article}
\usepackage[width=16cm]{geometry}                % See geometry.pdf to learn the layout options. There are lots.
\geometry{a4paper}                   % ... or a4paper or a5paper or ... 
%\geometry{landscape}                % Activate for for rotated page geometry
%\usepackage[parfill]{parskip}    % Activate to begin paragraphs with an empty line rather than an indent
%\usepackage{graphinlcx}
\usepackage{amssymb}
\usepackage{amsmath}
\usepackage{aliases}
\usepackage{color}
\usepackage{url}

\usepackage{listings}
\usepackage{cancel}
\usepackage{textcomp}

\lstset{
   language=matlab,
   keywordstyle=\bfseries\ttfamily\color[rgb]{0,0,1},
   identifierstyle=\ttfamily,
   commentstyle=\color[rgb]{0.133,0.545,0.133},
   stringstyle=\ttfamily\color[rgb]{0.627,0.126,0.941},
   showstringspaces=false,
   basicstyle=\small,
   numberstyle=\footnotesize,
   numbers=none,
   stepnumber=1,
   numbersep=10pt,
   tabsize=2,
   breaklines=true,
   prebreak = \raisebox{0ex}[0ex][0ex]{\ensuremath{\hookleftarrow}},
   breakatwhitespace=false,
   aboveskip={0.1\baselineskip},
    columns=fixed,
    upquote=true,
    extendedchars=true,
% frame=single,
    backgroundcolor=\color[rgb]{0.9,0.9,0.9}
}

\title{One dimensional Burgers' equation}
\author{Deep Ray, Ritesh Kumar, Praveen. C, Mythily Ramaswamy, J.-P. Raymond}
\date{}

\begin{document}

\maketitle
\begin{center}
IFCAM Summer School on Numerics and Control of PDE\\
22 July - 2 August 2013\\
IISc, Bangalore\\
\url{http://praveen.cfdlab.net/teaching/control2013}
\end{center}

%------------------------------------------------------------------------------
\section{Non-linear Burgers' equation}
We consider the non-linear viscous Burgers' equation
\begin{equation}
w_t -\nu w_{xx} + w w_x = f_s \qquad \mbox{in} \quad (0,1)\times (0,\infty)
\end{equation}
with boundary conditions
\begin{equation}
w(0,t) = u_s, \qquad w_x(1,t) = g_s
\end{equation}
and initial condition
\begin{equation}
w(x,0) = w_0(x)
\end{equation}

%-----------------------------------------------------------------------------------------
\subsection{Stationary Burgers' equation}
The stationary Burgers' equation is given by
\begin{equation}
-\nu w_{xx} + w w_x = f_s \qquad \mbox{in} \quad (0,1) 
\end{equation}
with boundary conditions
\begin{equation}
w(0) = u_s, \qquad w_x(1) = g_s
\end{equation}
An unstable stationary solution can be obtained as
\begin{equation}
 w_s = - \frac{\nu \pi}{2} (1 + \epsilon) \tan{\left( \frac{\pi}{4}(1+ \epsilon) + C_0 \right)}
\end{equation}
with the conditions
\[
 0 < \epsilon < 1, \qquad -\frac{\pi}{4} < C_0 < 0, \qquad (1 + \epsilon) \tan{\left( \frac{\pi}{4}(1+ \epsilon) + C_0 \right)} = 1
\]
We choose 
\[
\epsilon = 0.6 \qquad \Longrightarrow \qquad C_0 = \arctan(1/(1+\epsilon)) - \frac{\pi}{4}(1 + \epsilon) = -0.69803774609235
\]

% %-----------------------------------------------------------------------------------------
% \section{Nonlinear Burgers' equation without control}
% \begin{equation}
% w_t -\nu w_{xx} + w w_x = f_s
% \end{equation}
% with boundary conditions
% \begin{equation}
% w(0,t) = u_s, \qquad w_x(1,t) = g_s
% \end{equation}
% and initial conditions
% \begin{equation}
% w(x,0) = w_0(x) = w_s(x) + w^\delta(x)
% \end{equation}
% As done above, we multiply by a test function $\phi$ with $\phi(0)=0$ to obtain the weak formulation
% \[
% \int_0^1 w_t \phi \ud x + \nu \int_0^1 \dd{w}{x} \dd{\phi}{x} \ud x + \int_0^1 w \dd{w}{x} \phi \ud x = \int_0^1 f_s \phi \ud x + \nu g_s \phi(1)
% \]
% Divide domain into $N$ elements with vertices
% \[
% 0= x_0 < x_1 < \ldots < x_N = 1, \qquad x_i = ih, \qquad h = \frac{1}{N}
% \]
% The finite element solution can be written as
% \[
% w(x,t) = \sum_{j=1}^N w_j(t) \phi_j(x) + u_s \phi_0(x)
% \]
% while the stationary solution expressed as
% \[
% w_s(x) = \sum_{j=1}^N w_s^j \phi_j(x) + u_s \phi_0(x)
% \]
% Here the $\{ \phi_j \}$ are the usual $P_1$ basis functions. The Galerkin finite element method is
% \[
% \int_0^1 w_t \phi_i \ud x + \nu \int_0^1 \dd{w}{x} \dd{\phi_i}{x} \ud x + \int_0^1 w \dd{w}{x} \phi_i \ud x = \int_0^1 f_s \phi_i \ud x + \nu g_s \phi_i(1), \qquad i=1,2,\ldots,N
% \]
% 
% \[
% f_s = \sum_{j=0}^N f_j \phi_j
% \]
% 
% \begin{eqnarray*}
% && \sum_{j=1}^N w_j'(t) \int_0^1 \phi_j \phi_i \ud x+ \nu \sum_{j=1}^N w_j(t) \int_0^1 \phi_i' \phi_j' \ud x + \sum_{j=1}^N \sum_{k=1}^N w_j(t) w_k(t) \int_0^1 \phi_i \phi_j \phi_k' \ud x \\
% && + u_s \sum_{j=1}^N w_j(t) \left[ \int_0^1 \phi_i \phi_0 \phi_j' \ud x +  \int_0^1 \phi_i \phi_j \phi_0' \ud x \right] + u_s^2 \int_0^1 \phi_i \phi_0 \phi_0' \ud x + \nu u_s \int_0^1 \phi_i' \phi_0' \ud x\\
% &=& \int_0^1 f_s \phi_i \ud x + \nu g_s \phi_i(1), \qquad i=1,2,\ldots,N
% \end{eqnarray*}
% or
% \begin{eqnarray*}
% && \sum_{j=1}^N w_j'(t) \int_0^1 \phi_j \phi_i \ud x+ \sum_{j=1}^N w_j(t) \left [ \nu \int_0^1 \phi_i' \phi_j' \ud x + u_s \int_0^1 \phi_i (\phi_0 \phi_j)' \ud x \right ] \\
% && + \sum_{j=1}^N \sum_{k=1}^N w_j(t) w_k(t) \int_0^1 \phi_j \phi_k' \phi_i \ud x \\
% &=&  - u_s^2 \int_0^1 \phi_i \phi_0 \phi_0' \ud x - \nu u_s \int_0^1 \phi_i' \phi_0' \ud x + \int_0^1 f_s \phi_i \ud x + \nu g_s \phi_i(1), \qquad i=1,2,\ldots,N
% \end{eqnarray*}
%  We use the following notations:
% \begin{itemize}
%  \item $\w = [w_1,w_2,...,w_N]^\top$
%  \item $\M \in \re^{N \times N}$ with $\M(i,j) = \int_0^1 \phi_i\phi_j \ud x$
%  \item $\A1 \in \re^{N \times N}$ with $\A1(i,j) = \int_0^1 \phi_i'\phi_j' \ud x$
%  \item $\A2 \in \re^{N \times N}$ with $\A2(i,j) = \int_0^1 \phi_i(\phi_0 \phi_j)' \ud x$
%  \item $\A = \nu \A1 + u_s \A2$
%  \item $\D1 \in \re^{N \times N \times N}$ with $\D1(i,j,k) = \int_0^1 \phi_i \phi_j' \phi_k \ud x$
%  \item $\bdd1 \in \re^{N}$ with $\bdd1(i) = \int_0^1 \phi_i \phi_0 \phi_0' \ud x$
%  \item $\bdd2 \in \re^{N}$ with $\bdd2(i) = \int_0^1 \phi_i'\phi_0' \ud x$
%  \item $\bdd3 \in \re^{N}$ with $\bdd3(i) = \int_0^1 f_s \phi_i \ud x$
%  \item $\bdd4 = [\phi_1(1), \phi_2(1), ..., \phi_N(1)]^\top \in \re^{N}$
%  \item $\bdd = -u_s^2 \bdd1 - \nu u_s \bdd2 + \bdd3 + \nu g_s \bdd4 $
% \end{itemize}
% Thus we have a system of first order system of ODEs in $\w$ which we solve using forward Euler
% \begin{eqnarray*}
%  \M \w^{n+1} = \textbf{ RHS} 
% \end{eqnarray*}
% where 
% \begin{eqnarray*}
%  \textbf{RHS}(i) = \M(i,:) \w^n + dt \left[ - A(i,:) \w^n - (W^n)^\top \D1(:,:,i) \w^n  + \bdd \right]
% \end{eqnarray*}
% 
% %-----------------------------------------------------------------------------------------
\section{Nonlinear system without control}
We set $z = w - w_s$. The equation satisfied by $z$ is
\begin{equation}
\df{z}{t} + w_s \df{z}{x} + z \dd{w_s}{x} + z \dd{z}{x} = \nu \df{^2 z}{x^2} \qquad \mbox{in} \quad (0,1)\times (0,\infty)
\end{equation}
with boundary conditions
\begin{equation}
z(0,t) = 0, \qquad z_x(1,t) = 0
\end{equation}
and initial conditions
\begin{equation}
z(x,0) = w_0(x) - w_s(x)
\end{equation}
We multiply by a test function $\phi$ with $\phi(0)=0$ to obtain the weak formulation
\[
\int_0^1 z_t \phi \ud x + \nu \int_0^1 \df{z}{x} \dd{\phi}{x} \ud x + \int_0^1 w_s \df{z}{x} \phi \ud x + \int_0^1 z \df{w_s}{x} \phi \ud x + \int_0^1 z \dd{z}{x} \phi \ud x = 0
\]
Divide domain into $N$ elements with vertices
\[
0= x_0 < x_1 < \ldots < x_N = 1, \qquad x_i = ih, \qquad h = \frac{1}{N}
\]
The finite element solution can be written as
\[
z(x,t) = \sum_{j=1}^N z_j(t) \phi_j(x)
\]
while the stationary solution expressed as
\[
w_s(x) = \sum_{j=1}^N w_s^j \phi_j(x) + u_s \phi_0(x)
\]
where $w_s^j = w_s(x_j)$. Here the $\{ \phi_j \}$ are the usual $P_1$ basis functions. The approximate weak formulation is
\begin{eqnarray*}
\int_0^1 z_t \phi_i \ud x &=& - \nu \int_0^1 \df{z}{x} \dd{\phi_i}{x} \ud x - \int_0^1 w_s \df{z}{x} \phi_i \ud x - \int_0^1 z \df{w_s}{x} \phi_i \ud x \\
&& - \int_0^1 z \dd{z}{x} \phi_i \ud x, \qquad \forall i = 1,2,...,N 
\end{eqnarray*}
or
\begin{eqnarray*}
\sum_{j=1}^N z_j'(t) \int_0^1 \phi_i \phi_j \ud x &=& - \sum_{j=1}^N \sum_{k=1}^N z_j(t) z_k(t) \int_0^1 \phi_j \phi_k' \phi_i  \ud x \\
&& - \sum_{j=1}^N z_j(t) \left [ \nu  \int_0^1 \phi_i' \phi_j' \ud x + u_s \int_0^1 \phi_i (\phi_0 \phi_j)' \ud x  \right ] \\
&& - \sum_{j=1}^N z_j(t) \left [  \sum_{k=1}^N w_s^k(t) \left ( \int_0^1 \phi_j \phi_k' \phi_i \ud x +  \int_0^1 \phi_k \phi_j' \phi_i \ud x \right ) \right]
\end{eqnarray*}
 We use the following notations:
\begin{itemize}
 \item $\z = [z_1,z_2,...,z_N]^\top$
 \item $\w_s = [w_s^1,w_s^2,...,w_s^N]^\top$
 \item $\M \in \re^{N \times N}$ with $\M(i,j) = \int_0^1 \phi_i\phi_j \ud x$
 \item $\A1 \in \re^{N \times N}$ with $\A1(i,j) = \int_0^1 \phi_i'\phi_j' \ud x$
 \item $\A2 \in \re^{N \times N}$ with $\A2(i,j) = \int_0^1 \phi_i(\phi_0 \phi_j)' \ud x$
 \item $\D1 \in \re^{N \times N \times N}$ with $\D1(i,j,k) = \int_0^1 \phi_i \phi_j' \phi_k \ud x$
\end{itemize}
Thus we have a non-linear system of first order ODEs in $\z$
\begin{eqnarray*}
 \M \dd{\z}{t} = \textbf{ RHS} 
\end{eqnarray*}
where 
\begin{eqnarray*}
 \textbf{RHS}(i) = - \z^\top \D1(:,:,i) \z - \left[ \nu \A1(i,:) + u_s \A2(i,:) + \w_s^\top (\D1(:,:,i) + \D1(:,:,i)^\top) \right] \z
\end{eqnarray*}
This is implemented and solved in {\tt burger.m}
%-----------------------------------------------------------------------------------------

\paragraph{Warning} It is not good computational practice to compute the three dimensional matrices like $\D1$ since it requires large storage. But we will use this approach here since we are only in 1-D case.
%-----------------------------------------------------------------------------------------

\subsection{Excercises}

\begin{enumerate}

\item The initial condition is chosen as
\[
z(x,0) = \delta \sin(\pi x)
\]
Run program {\tt burger.m} with {\tt delta = 0}. The zero solution does not change with time. This corresponds to the stationary solution for the original Burgers's equation.

\item Take different initial conditions by varying {\tt delta}, and convince yourself that the zero solution is unstable. Try {\tt delta = 0.1} and {\tt delta = -0.1} and see how the solution evolves.
\end{enumerate}

%-----------------------------------------------------------------------------------------

\section{Linearized Burgers' equation}

\subsection{Feedback control}
Consider the linearized equation
\begin{equation}
\df{z}{t} + w_s \df{z}{x} + z \dd{w_s}{x} = \nu \df{^2 z}{x^2}
\end{equation}
\begin{equation}
z(0,t) = u(t), \qquad \df{z}{x}(1,t) = 0
\end{equation}
\begin{equation}
z(x,0) = w_0(x) - w_s(x)
\end{equation}
The weak formulation is given by
\begin{equation*}
\int_0^1 \left( \df{z}{t} + w_s \df{z}{x} + z \dd{w_s}{x} \right) \phi \ud x = - \nu \int_0^1 \df{z}{x} \df{\phi}{x} \ud x
\end{equation*}
We look for a finite element solution of the form
\[
z(x,t) = \sum_{j=1}^N z_j(t) \phi_j(x) + u(t) \phi_0(x)
\]
The approximate weak formulation is 
\begin{equation*}
\int_0^1 \left( \df{z}{t} + w_s \df{z}{x} + z \dd{w_s}{x} \right) \phi_i \ud x = - \nu \int_0^1 \df{z}{x} \df{\phi_i}{x} \ud x, \qquad i=1,2,\ldots,N
\end{equation*}
Ignoring the $\dd{u}{t}$ term gives us
\begin{eqnarray*}
\sum_{j=1}^N \dd{z_j}{t} \int_0^1 \phi_i \phi_j \ud x &=&
\sum_{j=1}^N z_j \left[ -\nu \int_0^1 \phi_i' \phi_j' \ud x - \int_0^1 w_s \phi_i \phi_j' \ud x - \int_0^1 \phi_i \phi_j \dd{w_s}{x} \right] \\
&& + u \left[ - \int_0^1 w_s \phi_i \phi_0' \ud x - \int_0^1 \phi_i \phi_0 \dd{w_s}{x} \ud x  - \nu \int_0^1 \phi_i' \phi_0' \ud x \right ]
\end{eqnarray*}
Using the expression for $w_s$ mentioned above, we get
\begin{eqnarray*}
\sum_{j=1}^N \dd{z_j}{t} \int_0^1 \phi_i \phi_j \ud x &=&
\sum_{j=1}^N z_j \left[ -\nu \int_0^1 \phi_i' \phi_j' \ud x - u_s \int_0^1 \phi_i (\phi_0 \phi_j)' \ud x\right] \\
&& + \sum_{j=1}^N z_j \left[ - \sum_{k=1}^N  w_s^k\int_0^1 \phi_j \phi_k' \phi_i \ud x - \sum_{k=1}^N  w_s^k\int_0^1 \phi_k \phi_j' \phi_i \ud x\right ] \\
&& + u \left[ - 2 u_s \int_0^1 \phi_i \phi_0 \phi_0' \ud x  - \sum_{k=1}^N w_s^k \int_0^1 \phi_i (\phi_0 \phi_k)' \ud x  - \nu \int_0^1 \phi_i' \phi_0' \ud x \right ]
\end{eqnarray*}
 We use the following notations:
\begin{itemize}
 \item $\z = [z_1,z_2,...,z_N]^\top$
 \item $\w_s = [w_s^1,w_s^2,...,w_s^N]^\top$
 \item $\M \in \re^{N \times N}$ with $\M(i,j) = \int_0^1 \phi_i\phi_j \ud x$
 \item $\A1 \in \re^{N \times N}$ with $\A1(i,j) = \int_0^1 \phi_i'\phi_j' \ud x$
 \item $\A2 \in \re^{N \times N}$ with $\A2(i,j) = \int_0^1 \phi_i(\phi_0 \phi_j)' \ud x$
 \item $\D1 \in \re^{N \times N \times N}$ with $\D1(i,j,k) = \int_0^1 \phi_i \phi_j' \phi_k \ud x$
 \item $\A(i,:) = - \nu \A1(i,:) -u_s \A2(i,:) - \w_s^\top \left[ \D1(:,:,i) + \D1(:,:,i)^\top \right], \quad \forall i = 1,2,...,N $
 \item $\bdd1 \in \re^{N}$ with $\bdd1(i) = \int_0^1 \phi_i \phi_0 \phi_0' \ud x$
 \item $\bdd2 \in \re^{N}$ with $\bdd2(i) = \int_0^1 \phi_i'\phi_0' \ud x$
 \item $\B = -2 u_s \bdd1 - \A2 \cdot \w_s - \nu \bdd2$
\end{itemize}
This can be written in matrix form as
\[
\M \dd{\z}{t} = \A \z + \B u
\]
The control $u(t)$ is obtained in terms of the feedback matix $\K = [K_1, K_2, ..., K_N]$ 
\[
 u(t) = - \sum_{j=1}^N K_j z_j(t)
\]
The feedback gain $\K$ is given by
\[
\K = \R^{-1} \B^\top \bdP \M
\]
where $\bdP$ is solution of algebraic Riccati equation (ARE)
\[
\A^\top \bdP \M + \M^\top \bdP \A - \M^\top \bdP \B \R^{-1} \B^\top \bdP \M + \Q  = 0
\]
\[
 (\M,\A - \B \K) \text{ is stable} 
\]

%------------------------------------------------------------------------------------------------------------
\subsection{Partial information with noise}
Consider the linearized model with noise in the model
\begin{equation}
\df{z}{t} + w_s \df{z}{x} + z \dd{w_s}{x} = \nu \df{^2 z}{x^2} + \eta
\end{equation}
\begin{equation}
z(0,t) = u(t), \qquad \df{z}{x}(1,t) = 0
\end{equation}
\begin{equation}
z(x,0) = w_0(x) - w_s(x)
\end{equation}
Assume that we have access to partial information corrupted by noise
\[
y = Hz + \mu
\]
$H$ being a suitable observation operator. We shall consider the case where $H$ is given by
\[
 Hz(t) = z(1,t)
\]
In the FEM setup, we get
\[
 \M \dd{\z}{t} = \A \z + \B u + \boldsymbol\eta 
\]
\[
 \y = \bH \z + \boldsymbol\mu
\]
where the observation operator is approximated as
\[
 \bH = (0,0,...,0,1)^\top \in \re^{N}
\]


%--------------------------------------------------------------------------------

\subsection{Estimation problem}
Consider the 
\begin{equation}
\df{z_e}{t} + w_s \df{z_e}{x} + z_e \dd{w_s}{x} = \nu \df{^2 z_e}{x^2} + L( H z - H z_e )
\end{equation}
\begin{equation}
z_e(0,t) = u(t), \qquad \df{z_e}{x}(1,t) = 0
\end{equation}
\begin{equation}
z_e(x,0) = 0
\end{equation}
In the FEM setup, we have
\[
\M \dd{\z_e}{t} = \A \z_e + \B u + \bL(\y - \bH \z_e)
\]
The filtering gain $\bL$ is given by 
\[
\bL = \M \bdP_e \bH^\top \R^{-1}_{\boldsymbol\mu}
\]
where $\bdP_e$ is solution of
\[
\A \bdP_e \M + \M \bdP_e \A^\top - \M \bdP_e \bH^\top \R_{\boldsymbol\mu}^{-1}  \bH \bdP_e \M + \R_{\boldsymbol\eta}  = 0
\]
\[
 (\M,\A - \bL \bH) \text{ is stable} 
\]

%--------------------------------------------------------------------------------

\subsection{Coupled linear system}
The feedback is based on estimated solution $u = -\K \z_e$
\begin{eqnarray*}
\M \dd{\z}{t} &=& \A \z - \B \K \z_e + \boldsymbol\eta \\
\M \dd{\z_e}{t} &=& \bL \bH \z + (\A - \B \K - \bL \bH) \z_e + \bL \boldsymbol\mu
\end{eqnarray*}
or in matrix form
\[\begin{bmatrix}
   \M & 0\\
   0 & \M \\
  \end{bmatrix}
\dd{}{t} \begin{bmatrix}
\z \\
\z_e \end{bmatrix} = \begin{bmatrix}
\A & -\B \K \\
\bL \bH & \A - \B \K - \bL \bH \end{bmatrix} \begin{bmatrix}
\z \\ \z_e \end{bmatrix} + \begin{bmatrix}
\I & 0 \\
0 & \bL \end{bmatrix} \begin{bmatrix}
\boldsymbol\eta \\ \boldsymbol\mu \end{bmatrix}
\]
The initial condition is given by
\[
\z(0) = \z_0, \qquad \z_e(0) = 0
\]
This is implemented in program {\tt lin\_lqg.m}

\subsection{Excercises}

\begin{enumerate}

\item Run program {\tt lin\_lqg.m}

\item Check stabilizabilty of $(\A,\B)$ using the Hautus criterion.

\item Set $(\Q = 0)$ which corresponds to minimal norm control. Run the code and observe the solution and control.

\item Set $(\Q = \M)$ and run the code. How has the solution and control beaviour changed~?

\item Vary $\R$ in the range $(0.01,10)$ and observe how the feedback gain $\K$ varies.

\item For $\Q = 0$, modify the code to solve the control and estimation problem for only the unstable components.
\end{enumerate}


%-------------------------------------------------------------------------------------------------------------
\section{Non-linear system with control and estimation}
We will solve the non-linear equation for the perturbation $z = w - w_s$.
\begin{equation*}
\df{z}{t} + w_s \df{z}{x} + z \dd{w_s}{x} + z \dd{z}{x} = \nu \df{^2 z}{x^2}
\end{equation*}
\begin{equation*}
z(0,t) = u(t), \quad z_x(1,t) = 0,  \quad z(x,0) = w^\delta(x)
\end{equation*}
We multiply by a test function $\phi$ with $\phi(0)=0$ to obtain the weak formulation
\[
\int_0^1 z_t \phi \ud x + \nu \int_0^1 \df{z}{x} \dd{\phi}{x} \ud x + \int_0^1 w_s \df{z}{x} \phi \ud x + \int_0^1 z \df{w_s}{x} \phi \ud x + \int_0^1 w \dd{w}{x} \phi \ud x = 0
\]
The finite element solution can be written as
\[
z(x,t) = \sum_{j=1}^N z_j(t) \phi_j(x) + u(t) \phi_0(x)
\]
while the stationary solution expressed as
\[
w_s(x) = \sum_{j=1}^N w_s^j \phi_j(x) + u_s \phi_0(x)
\]
The Galerkin approximation is
\begin{eqnarray*}
\int_0^1 z_t \phi_i \ud x &=& - \nu \int_0^1 \df{z}{x} \dd{\phi_i}{x} \ud x - \int_0^1 w_s \df{z}{x} \phi_i \ud x - \int_0^1 z \df{w_s}{x} \phi_i \ud x \\
&& - \int_0^1 z \dd{z}{x} \phi_i \ud x, \qquad \forall i=1,2,...,N
\end{eqnarray*}
Ignoring the $\dd{u}{t}$ term and using the expression for $w_s$, we get
\begin{eqnarray*}
\sum_{j=1}^N z_j'(t) \int_0^1 \phi_i \phi_j \ud x &=& - \sum_{j=1}^N \sum_{k=1}^N z_j(t) z_k(t) \int_0^1 \phi_j \phi_k' \phi_i  \ud x \\ 
&& - \sum_{j=1}^N z_j(t) \left [ \nu  \int_0^1 \phi_i' \phi_j' \ud x + u_s \int_0^1 \phi_i (\phi_0 \phi_j)' \ud x  \right ] \\
&& - \sum_{j=1}^N z_j(t) \left [  \sum_{k=1}^N w_s^k(t) \left ( \int_0^1 \phi_j \phi_k' \phi_i \ud x +  \int_0^1 \phi_k \phi_j' \phi_i \ud x \right ) \right]\\
&& - u(t) \left [\sum_{j=1}^N z_j \int_0^1 \phi_i (\phi_0 \phi_j)' \ud x  + \sum_{k=1}^N w_s^k \int_0^1 \phi_i (\phi_0 \phi_k)' \ud x \right ]\\
&& -  u(t) \left[ \nu \int_0^1 \phi_i' \phi_0' \ud x + 2 u_s \int_0^1 \phi_i \phi_0 \phi_0' \ud x \right] - u(t)^2 \int_0^1 \phi_i \phi_0 \phi_0' \ud x
\end{eqnarray*}
Thus we have the system
\begin{eqnarray*}
 \M \dd{\z}{t} = \textbf{ RHS}_1 + \textbf{ RHS}_2 
\end{eqnarray*}
where 
\begin{eqnarray*}
 \textbf{RHS}_1(i) &=& - \z^\top \D1(:,:,i) \z - \left[ \nu \A1(i,:) + u_s \A2(i,:) + \w_s^\top (\D1(:,:,i) + \D1(:,:,i)^\top) \right] \z \\
 \textbf{RHS}_2(i) &=& - u(t) \left[ \A2 (\z + \w_s) + 2 u_s \bdd1 +\nu \bdd2 \right] - u(t)^2 \bdd1 
\end{eqnarray*}
The control $u(t)$ is obtained from the linear estimation problem
\[
 u(t) = - \sum_{j=1}^N K_j z_e^j
\]
We solve the non-linear system and the linear estimator in a coupled manner.
\begin{eqnarray*}
 \M \dd{\z}{t} &=& \textbf{ RHS}_1 + \textbf{ RHS}_2\\
 \M \dd{\z_e}{t} &=& \bL \bH \z + (\A - \B \K - \bL \bH) \z_e
\end{eqnarray*}
\[
\z(0) = \z_0, \qquad \z_e(0) = 0
\]
This is implemented in {\tt burger\_lqg.m}

\subsection{Excercises}

\begin{enumerate}

\item Run program {\tt burger\_lqg.m}

\item Vary $\R$ in the range $(0.01,10)$ and observe how the feedback gain $\K$ changes. How does the solution and control change with $\R$.

\end{enumerate}

\section{List of Programs}

\begin{enumerate}

\item {\tt get\_system\_mat.m}: Computes FEM matrices

\item {\tt feedback\_matrix.m}: Computes feedback matrix

\item {\tt stationarysol.m}: Computes exact unstable stationary solution

\item {\tt burger.m}: Solves the non-linear model, without feedback

\item {\tt lin\_lqg.m}: Solves the coupled estimation and control system for the linear problem

\item {\tt burger\_lqg.m}: Solves the coupled estimation and control system for the non-linear problem

\item {\tt rhs\_burger.m}: Computes the right hand side for the non-linear problem, without feedback

\item {\tt rhs\_burger\_lqg.m}: Computes the right hand side for coupled estimation and control system with the non-linear problem

\end{enumerate}
\end{document} 

